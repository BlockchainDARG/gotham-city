% This is samplepaper.tex, a sample chapter demonstrating the
% LLNCS macro package for Springer Computer Science proceedings;
% Version 2.20 of 2017/10/04
%
\documentclass[runningheads]{llncs}
\usepackage{mdframed}
\usepackage{lipsum}
\usepackage{graphicx,algorithm,algpseudocode}
%
\usepackage{graphicx}
% Used for displaying a sample figure. If possible, figure files should
% be included in EPS format.
%
% If you use the hyperref package, please uncomment the following line
% to display URLs in blue roman font according to Springer's eBook style:
% \renewcommand\UrlFont{\color{blue}\rmfamily}

\begin{document}
%
\title{Bitcoin Wallet Powered by Two Party ECDSA - Extended Abstract}
%
%\titlerunning{Abbreviated paper title}
% If the paper title is too long for the running head, you can set
% an abbreviated paper title here
%
\author{Team KZen}
%
% First names are abbreviated in the running head.
% If there are more than two authors, 'et al.' is used.
%
\institute{https://github.com/KZen-networks }
%
\maketitle              % typeset the header of the contribution
%
\begin{abstract}
We demonstrate a Bitcoin wallet that utilizes two party ECDSA (2P-ECDSA \cite{lindell17}). Our architecture consists of a simple client-server communication model. We show support for 2 party deterministic child derivation (2P-HD), secret share rotation and verifiable recovery. We discuss the opportunities and challenges of using a multi-party wallet.


\end{abstract}
%
%
%
\section{Background}
For end-users, cryptocurrencies and blockchain-based assets are hard to store and manage. One of the reasons is the tradeoff between security and availability. Storing private keys safely requires dedicated hardware or extreme security measures which make using the coins on a daily basis difficult. Threshold cryptography provides ways to decentralize the private key and digital signing. This can potentially benefit security but at the same time reveal new challenges such as availability, ownership and recovery.
Bitcoin is utilizing ECDSA as the signing scheme. There is an active line of research for practical and efficient multi-party ECDSA schemes \cite{mackenzie,boneh17,lindell17,doerner,goldfeder18,lindell18}.

\section{A General ThreshSig Wallet}
Informally, a threshold wallet is described by the following algorithms:
\begin{itemize}
    \item \textbf{Distributed Key Generation (DKG):} a multi-party algorithm to generate a public key (a blockchain address) and secret shares of the private key.
    \item \textbf{Distributed Signing:} a multi-party algorithm that takes the DKG secret shares and additional public data to generate a valid signature on a given message for the DKG public key (a blockchain transaction).
    \item \textbf{Deterministic Child Address Derivation:} a multi-party algorithm to deterministically derive new public key and secret shares from public key and secret shares.
    \item \textbf{Rotation:} A multi-party protocol to rotate the secret shares without changing the public key.
    \item \textbf{Secret Shares Recovery:} a protocol to recover a secret share.
\end{itemize}
We state that in a threshold wallet at no point in time one party will hold the full private key. All multi-party protocols must be secure against malicious adversaries. \\
A wallet requires also read and write access to a full node. We do not define it here but emphasize that there is no privacy between parties running a threshold wallet. It is recommended that as many parties as possible will have independent access to the blockchain network. A party without a trusted connection to blockchain will run additional protocol to authenticate the message to be signed.
\subsection{Comparison to Multisig wallet}
We can learn on the properties and advantages of using threshold signatures (ThreshSig) by comparing to MultiSig. Multisig is the current way to add threshold security to private key management. Key rotation means that an attacker must attack simultaneously different parties to get access to the full private key. In ThreshSig wallet this is done without changing the public key and without going through the blockchain. In MultiSig wallet it can be done only by making a blockchain transaction which costs time and money and change the public key as well. Referring to bitcoin blockchain, a ThreshSig transaction (tx) is indistinguishable from a regular while Multisig tx has a more unique structure that can attest to the MultiSig policy for other observers of the network. The cost of a ThreshSig tx is again the same as regulat tx since they have the same size. A MultiSig tx on the other hand, has a size and therefore cost proportional to the number of participating parties. Finally, ThreshSig can be easily adapted for different types of blockchains as it depends only on cryptography. MultiSig is relying on the application layer of a blockchain , i.e. Script in Bitcoin, smart contract in Ethereum and therefore requires more effort to add support for new chains.

\section{Design Considerations for Choosing 2P-ECDSA}
 We define two roles: Owners and Providers. An Owner is the end user who owns the funds in the account and holds one secret share of the private key. The Provider is another share holder of the private key but has no funds tied to this private key. His role is to provide the additional security in the system, enabling the owner to generate keys and transact in distributed fashion. For maintaining maximal power at the hands of the Owner we have two minimal claims:
\begin{enumerate}
    \item No tx can be made without the Owner participation
    \item At any point in time the Owner can recover the full private key
\end{enumerate}
Working with 2 parties allows us to use classical server-client communication model. The Client will be the Owner and the Server will be the Provider. Working in a more general $\{t,n\}$ threshold model will have an efficiency penalty, will require p2p communication network with broadcast and will violate the first claim. Therefore we chose to work in a 2 party setting because of both simplicity and security.
\section{Implementation Details for 2P-Thresh-Wallet}
\subsection{\{2,2\} Key Generation}
In our setting the two parties that jointly compute the signature are a server and a client. Naturally the server is  has access to more resources and will take the role in the protocols of the party that requires more computing power. In \cite{lindell17} terminology the server will play the role of party ${P_1}$ wheres the client will play the role of party ${P_2}$.

\begin{mdframed}[userdefinedwidth = 12.3cm]
\textbf{\{2,2\} Key Gen algorithm}
\begin{enumerate}
\item Init: Server Chooses a random ${x_1}$  and computes ${Q_1 = x_1 \cdot G}$, Client Chooses a random ${x_2}$  and computes ${Q_2 = x_2 \cdot G}$
\item Server and Client run ECDH key exchange (in fact a bit more involved version of EC-DH \cite{lindell17}). $Q=x_1x_2G$ is the resulting joint public key.
\item The Server generates Paillier key-pair and computes ${c_{key} =Enc_{pk}(x_1)}$ where ${pk}$ is the Paillier public key. ${c_{key}}$ and ${pk}$ are sent to the client.
\item The Server and Client run a proof of correct Paillier public key. We use a non interactive version of the proof given in \cite{sharon} with parameters given in \cite{lindell18}. The Server is the prover and the client is the verifier.
\item The Client initiates a 2-round zero-knowledge protocol for the server to prove that the value encrypted in ${c_{key}}$ is the discrete log of ${Q_1}$. The protocol is given in \cite{lindell17} section 6.
\item Client initiates a 2-round zero-knowledge protocol for the server to prove that ${x_1 \in \mathbb{Z}_q}$ where ${q}$ is the order of the elliptic curve. The protocol is given in \cite{lindell17} Appendix A.
\end{enumerate}

\end{mdframed}

All commitments in the protocol are hash based. We use SHA256.
\subsection{\{2,2\} Signing}
We first describe ECDSA signature scheme: \\
 Assume that Alice has a private key ${x}$. the corresponding public key is ${x \cdot G}$ and she wants to sign a message ${m}$. Alice can compute a signature on a message ${m}$ as follows:
 \begin{enumerate}
 \item Choose ephemeral key ${k}$.
 \item Compute ${R = k \cdot G}$.
 \item Assume ${R}$ is the point ${(r_x, r_y)}$. Then, set ${r}$ to be the ${x}$ coordinate of the curve point ${R}$.
 \item Compute the signature: ${s = k^{-1} \cdot (H(m)+ r\cdot x)}$, where ${H}$ denotes hash function.
 \item Output ${(r,s)}$
 \end{enumerate}
Here is the equivalent two party signing algorithm that results in the same signature.
\begin{mdframed}[userdefinedwidth = 12.3cm]
\textbf{\{2,2\} Signing algorithm}
\begin{enumerate}

\item Server and client repeat step 1 of the Key Generation protocol but for ephemeral key-pairs: ${k_1,R_1}$ and ${k_2,R_2}$ for the server and client respectively.
\item Server computes ${R =k_1 \cdot R_2}$. Client computes ${R = k_2 \cdot R_1}$. From the same ${R}$ both can extract the x-coordinate ${r=r_x}$.
\item Client computes ${c_1=Enc_{pk}(k_2^{-1}\cdot m' + \rho q)}$ and ${c_2={c_{key}}^{x_2\cdot r \cdot k_2^{-1}}}$. Here ${\rho}$ is some random number. The client then computes and sends ${c_3 = c_1 \oplus c_2}$ where ${\oplus}$ denotes the additive homomorphic operation of Paillier cryptosystem.
\item Server decrypts ${c_3}$ to get ${s'}$. The server computes ${s=s'\cdot k_1^{-1}}$ and output ${(r,s)}$ as the ECDSA signature.
\end{enumerate}
\end{mdframed}


\subsection{2P-HD}
The purpose of 2P-HD protocol is to allow each party to hold one master key. The master key can derive all other keys and it is the one to be recovered if needed. This protocol achieves the security guarantees in the spirit of BIP32 but is incompatible to the standard. Instead it was designed to support natively the two party case with much efficiency. Forcing BIP32 will make an inefficient protocol and since there is no way to import or export between BIP39/BIP32 compatible wallet to a 2 party wallet it reduces a lot to benefit if supporting the standard.
\begin{mdframed}[userdefinedwidth = 12.3cm]
\textbf{2P-HD algorithm}
\begin{enumerate}
\item The Server and Client run ECDH to get a shared secret which we call chain code $cc$.
\begin{enumerate}
\item For a given $i$, Both parties compute :
\begin{enumerate}
\item ${f = \mathrm{HMAC}(key=cc, data = Q || i )}$
\item ${f=f_l||f_r}$ where ${|f_l|=|f_r|=256}$
\item ${Q_2' = f_l\cdot Q_2}$
\item ${cc' = f_r \cdot cc}$
\end{enumerate}
\item The Client updates his secret share to $x_2' = f_l \cdot x_2$
\end{enumerate}
\end{enumerate}
\end{mdframed}
The protocol takes as input a vector of indices and repeats the algorithm each time with new $Q$ and $i$ is the next index.

\subsection{2P-Rotation}
The secret shares are multiplicative, i.e. $Q=x_1x_2G$. Therefore, updating $x_1' = rx_1$ and $x_2 = r^{-1}x_2$ for random $r$ results in re-randomization of the secret shares with the same public key.
\begin{mdframed}[userdefinedwidth = 12.3cm]
\textbf{2P-Rotation}
\begin{enumerate}

\item Run a string coin toss protocol (we use optimal rounds version of coin toss from \cite{simulate}), the result is a random string $r$
\item Client updates $x_2' = r^{-1}x_2$ and $Q_1' = r*Q_1$.
\item Server generates a new Paillier key pair and encrypts $x_1' = rx_1$.
\item The Server an Client run the same zero knowledge proofs as in $\{2,2\}$ KeyGen, namely - proof of correct Paillier public key and PDL proof including range proof, using $Q_1'$ instead of $Q_1$
\end{enumerate}
\end{mdframed}
It is important to notice that 2P-HD and Rotation are commutative and can be applied in different order with the same outcome. This means that only the master key needs to be rotated.
\subsection{Recovery}
\subsubsection{Recovery} We use a protocol based on DLog Verifiable Encryption \cite{cs03}. The backup of a secret share is done with the help of an Escrow. Lets assume the Client wants the Server to backup his $x_1$. The escrow has a known public key $Q_e$ (might be attached to a blockchain address). The Server will send the Client an encryption $c$ of $x_1$ under the escrow public key. The Server will also send the Client a zero knowledge proof that the encryption is for a DLog corresponds to $Q_1$. The Escrow can periodically send a life signal that he still owns the private key $k_e$ corresponds to $Q_e$ by sending a signature with $k_e$ on a message from today's paper. When the Client needs a recovery, if for example the Server did not send a life signal to the Escrow (or some smart contract) for long time, the Escrow publish its private key and the Client can decrypt $c$ and get $x_1$. \\
Similarly, the Server can keep the encryption $c$ and use it a backup for $x_1$, such that if something happens to it, the Server can recover by asking the Escrow for its private key.
\begin{mdframed}[userdefinedwidth = 12.3cm]
\textbf{Verifiable Recovery}
\begin{enumerate}

\item \textit{Encryption:} The Server divides $x_1$ to $m$ small enough segments $[x_1]_i, i\in \{1,...,m\}$. Each segment is encrypted using the Escrow public key $Q_k$ (such that $Q_k = k_eG$) using ElGamal "encryption in the exponent" : $(D_i, E_i) = ([x_1]_iG+r_iQ_e, r_iG)$. The server sends the ciphertext to the Client
\item \textit{Proofs:} The server sends zk proofs: (1)  $x_1 = \sum_{i=1}^{m}{[x_1]_i}$ (2)  proof of knowledge that $(D_i, E_i) = ([x_1]_iG+r_iQ_e, r_iG)$ 3. proof that the segments bit size is small (we use Bulletproofs \cite{bulletproof}).
\item The Escrow sends periodically signature of a time-stamped message using $k_e$. The Client Verifies the signature using $Q_e$
\item \textit{Decryption:} After $k_e$ becomes public the user will decrypt segment $[x_1]_i$ from $D_i - k_eE_i$ by using best known algorithm for finding DLog over a small space of options.
\end{enumerate}
\end{mdframed}
We comment that a reciprocal treatment can be done for the Client backup.

\section{Code Architecture}
In general, we try to maximize the amount of code we re-use. Our stack is built completely in Rust. At the bottom we use secp256k1 elliptic curve (EC) library \cite{secp}. On top we have a utility library that implements simple EC cryptographic primitives \cite{curv}. At the same level we also wrote a library for zk proofs over Paillier cryptosystem \cite{zk-paillier} that uses another library for basic Paillier operations \cite{paillier}. On top we have our 2p-ecdsa library \cite{ecdsa} that uses Paillier and ECC that we implemented below. On top that we have a key management system (kms) library \cite{kms} that pack 2p-ecdsa into to optimize communication and adds the concept of master key with rotation, 2p-hd and recovry. Finally we have a wallet app with network, DB and full node access \cite{gotham}.

\section{Challenges}
Performance without compromising on security represents the main challenge. For a good user experience we need fast signing and key derivation times since this operations are reoccurring. Key Generation and backup are done once but still need to be reasonably fast. To achieve this we optimized on multiple fronts:  We minimized communication rounds by choosing zk proofs that can run non interactively (unless the communication tradeoff was worse). We aggregated different steps of the protocols into the same message whenever it was possible without harming security. We used software optimizations like parallelism of different Paillier encryptions. We used mathematical best practices like using Chinese Reminder Theorem for Paillier encryptions. Finally, we used native Rust all the way to not suffer when bridging between languages.

\begin{thebibliography}{8}
\bibitem{mackenzie}
P.D. MacKenzie and M.K. Reiter. Two-party generation of DSA signatures. International
Journal of Information Security,pp 218–239, (2004). An extended
abstract appeared at CRYPTO 2001

\bibitem{boneh17}
D. Boneh, R. Gennaro, S. Goldfeder. Using Level-1 Homomorphic Encryption To
Improve Threshold DSA Signatures For Bitcoin
Wallet Security. In Latincrypt 2017.
\bibitem{lindell17}
Y. Lindell. Fast Secure Two-Party ECDSA Signing.
In CRYPTO 2017, Springer (LNCS 10402), pages 613-644, 2017.
\bibitem{doerner}
J. Doerner, K, Yashvanth, L. Eysa, A, Shelat. Secure Two-party Threshold ECDSA from ECDSA Assumptions. In 2018 IEEE Symposium on Security and Privacy (SP), pp. 980-997. IEEE, 2018.
\bibitem{goldfeder18}
R. Gennaro, S. Goldfeder. Fast Multiparty Threshold ECDSA with Fast Trustless Setup
ACM Conference on Computer and Communications Security (CCS), 2018.
\bibitem{lindell18}
Y. Lindell and A. Nof. Fast Secure Multiparty ECDSA with Practical Distributed Key Generation and Applications to Cryptocurrency Custody.
ACM Conference on Computer and Communications Security (CCS), 2018.

\bibitem{sharon}
S. Goldberg, L. Reyzin, O. Sagga and F. Baldimtsi. Certifying RSA Public Keys with anEfficient NIZK. Cryptology ePrint Archive: Report 2018/057, 2018

\bibitem{simulate}
Y. Lindell, How to Simulate iT - A Tutorial on the Simulation Proof Technique, https://eprint.iacr.org/2016/046.pdf
\bibitem{cs03}
J. Camenisch, and V. Shoup. Practical verifiable encryption and decryption of discrete logarithms. Advances in Cryptology — CRYPTO 2003. pages 126-144, Springer, Berlin, Heidelberg, 2003.
\bibitem{bulletproof}
B. Bunz, J. Bootle, D. Boneh, A. Poelstra, P. Wuille, and G. Maxwell,
Bulletproofs: Short proofs for confidential transactions and more. In Security and Privacy(SP) IEEE Symposium, 2018.
\bibitem{secp}
https://github.com/rust-bitcoin/rust-secp256k1
\bibitem{curv}
https://github.com/KZen-networks/curv
\bibitem{zk-paillier}
https://github.com/KZen-networks/zk-paillier
\bibitem{paillier}
https://github.com/mortendahl/rust-paillier
\bibitem{ecdsa}
https://github.com/KZen-networks/multi-party-ecdsa
\bibitem{kms}
https://github.com/KZen-networks/kms-secp256k1
\bibitem{gotham}
https://github.com/KZen-networks/gotham-city
\end{thebibliography}
\end{document}
